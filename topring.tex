\documentclass[11pt,a4paper,fleqn]{article}
\usepackage{mathrsfs}
\usepackage{amsfonts}
\usepackage{amsmath}
\usepackage{analysis, algebra}%mypackage
\usepackage{amssymb}
\usepackage{enumerate}
\usepackage{graphicx,subfigure,float}
\usepackage{fancyhdr,times}
\usepackage{indentfirst}
\usepackage{hyperref}
%\usepackage{cite}
\usepackage[numbers,sort&compress]{natbib}

\hypersetup{
   colorlinks=true,%
   citecolor=green,%
   filecolor=black,%
   linkcolor=black,%
   urlcolor=blue
} %

\begin{document}

\renewcommand{\S}{\mathscr{S}}                       %
\renewcommand{\d}{\mathrm{d}}                          %
\renewcommand{\i}{\mathrm{i}}                          %
\newcommand{\ei}[1]{\mathrm{e}^{\mathrm{i}#1}}              %
\newcommand{\id}{\mathrm{id}}                    %%norm
\newcommand{\gen}[1]{\langle#1\rangle}                                    %
\newcommand{\fix}{\mathrm{fix}}                               %  
\newcommand{\Her}{\mathrm{Her}}                               %       %
\newcommand{\map}{\rightarrow}                               % %
\newcommand{\Rxt}{{\textbf{Rxt}}}                              %
\newcommand{\Fld}{{\textbf{Fld}}}                              %
\newcommand{\Dom}{{\textbf{Dom}}}                              %
\renewcommand{\M}{{\mathrm{M}}}
\newcommand{\Prm}{{\mathrm{P}}}
\newcommand{\TopRing}{{\mathrm{TopRing}}}

\newenvironment{proof}{\emph{Proof.}\hspace{0.5cm}}{$\hspace{0.4cm}\Box$\\}
\newtheorem{theorem}{Theorem}[section]
\newtheorem{lemma}{Lemma}[section]
\newtheorem{remark}{Remark}[section]
\newtheorem{corollary}{Corollary}[section]
\newtheorem{fact}{Fact}[section]
\newtheorem{example}{Example}[section]
\newtheorem{definition}{Definition}[section]

%\footnotetext
\bibliographystyle{plain}  % ������˳������


\title{Topological Ring Theory}
\date{}
\author{S. William}
\maketitle

\section{Basic Concepts}
We show the definition of topological rings.
\begin{definition}[Topological Ring]\label{df:topring}
A ring $R$ with a topological structure is a topological ring if the operators $-, *:R\times R\to R$ are continuous. It is separated, when its topology is separated (Hausdorff space).
\end{definition}

We show some obvious facts.
\begin{fact}
\begin{enumerate}
 \item
$R$ is an Abelian group with $+$;
 \item
 A subring $M$ and quotient ring $R/J$ of $R$ is also topological rings with topological structure inherited from $R$; $R/J$ is separated iff $J$ is closed;
 \item
 The closure $\bar{M}$ of a subring $M$ is also a topological ring. A direct product of topological rings is a topological ring in a natural way;
 \item For a separated ring, we define completion $\tilde{R}$ (also separated) of $R$ regarding $R$ as a uniform space. And we have an extension $\tilde{R}/R$.
\end{enumerate}
\end{fact}


\begin{example}
$\R$ is a topological ring (complete, separated), and $\R=\tilde{\Q}$. So $\C$ is.
\end{example}


\begin{example}
$\R^{n\times n}$ is a topological ring (complete, separated). So $\C^{n\times n}$ is.
\end{example}


\begin{definition}[$J$-adic topology]
 $J$ is an ideal of a communative tring $R$, then set $\{J^m, m\in\N\}$ forms a fundamental system of neighbourhoods of 0 that generates so-called $J$-adic topology. It is separated if $\bigcap_mJ^m=\{0\}$.
 \end{definition}

\begin{example}
 The $p$-adic topology on the integers is an example of an $(p)$-adic topology.
\end{example}


 $\TopRing$ denotes the category of topological rings with continuous homomorphisms as its morphisms.

 \section{Top on Homomorphism}

Homomorphism set $\Hom(R,S)$ is a ring, $\phi\psi(a)=\phi(a)\psi(a),(\phi+\psi)(a)=\phi(a)+\psi(a)$.
Let $M(A,V)=\set{\phi\in\Hom(R,S)}{\phi[A]\subset V}$. If $A\in\mathcal{G},V\in\mathcal{B}$, then $M(A,V)$ form a top basis of $\Hom(R,S)$, where $\mathcal{G}$ is directed (e.g. $\mathcal{G}=\mathcal{K}$). Therefore, $\Hom(R,S)$ is a top ring.

\bibliographystyle{unsrturl} % Style BST file
\bibliography{refer}      % Bibliography file (usually '*.bib' )
\end{document}
