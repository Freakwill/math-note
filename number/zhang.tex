\documentclass[11pt,a4paper,fleqn]{article}
\usepackage{mathrsfs,amsfonts,amsmath, amssymb}
\usepackage{analysis}%mypackage
\usepackage{logic}

\usepackage{enumerate}
\usepackage{graphicx,subfigure,float}
\usepackage{fancyhdr,times}
\usepackage{indentfirst}
\usepackage{hyperref}
%\usepackage{cite}
\usepackage[numbers,sort&compress]{natbib}

\hypersetup{
   colorlinks=true,%
   citecolor=green,%
   filecolor=black,%
   linkcolor=black,%
   urlcolor=blue
} % 


\newcommand{\ned}{\not|}

\newenvironment{proof}{\emph{Proof.}\hspace{0.5cm}}{$\hspace{0.4cm}\Box$\\}
\newtheorem{theorem}{Theorem}[section]
\newtheorem{lemma}{Lemma}[section]
\newtheorem{remark}{Remark}[section]
\newtheorem{corollary}{Corollary}[section]
\newtheorem{conjecture}{Conjecture}[section]
\newtheorem{example}{Example}[section]
\newtheorem{definition}{Definition}[section]

%\footnotetext
\bibliographystyle{plain}  % °´²åÈë˳ÐòÅÅÁÐ


\begin{document}
\title{Report of Zhang Yitang}
\date{}
\author{S. William}
\maketitle


Let $P$ be the set of all prime numbers.
\begin{definition}
  $\{h_i,i=1,\cdots, k\}\subset\N$ is admissible iff
  \[\forall p\in P\exists n\in\N, (P(n)=\prod_i(h_i+n),p)=1 ~ (p\ned h_i+n,i=1,\cdots, k).\]
\end{definition}

\begin{example}
 $\{0,2\},\{0,2,6\}$ is admissible. $\{h,h+1\},\{h,h+2,h+4\}$ is not admissible.
\end{example}

\begin{lemma}
  $\{h_i,i=1,\cdots k\}\subset\N$ is admissible if
  \[\exists^\infty n, h_i+n\in P,i=1,\cdots, k.\]
\end{lemma}

\begin{conjecture}[Hardy-Littlewood Conjecture]
 If $\{h_i,i=1,\cdots k\}\subset\N$ is admissible, then
  \[\exists^\infty n, h_i+n\in P,i=1,\cdots, k.\]
\end{conjecture}

\begin{theorem}\label{th:zhang}
  If $\{h_i,i=1,\cdots k\}\subset\N$ is admissible, then
  \[\lrg k \exists^\infty n \exists^2 i, h_i+n\in P,i=1,\cdots, k.\]
\end{theorem}

\begin{proof}
  Use Lemma \ref{lm:zhang}.
\end{proof}

\begin{definition}Define an arithmetic function
  $\theta(x)=\begin{cases}
    \ln x, & x\in P,\\
    0, & x\not\in P.
  \end{cases}$
\end{definition}

\begin{lemma}\label{lm:zhang}
For large $x$,
  \[\exists n\sim x (x\leq n< 2x),\sum_i\theta(n+h_i) > \ln 3x.\]
\end{lemma}
\begin{proof}
  Let $f(n)\geq 0$. 引入$S_1=\sum_{n\sim x}f(n),S_1=\sum_{n\sim x}(\sum_i\theta(n+h_i))f(n)$. Only need prove
  \[S_2-S_1\ln 3x >0.\]
  Let $D=x^c,c\in(0,\inv{2})$. Choose $f(n)=\lambda(n)^2$, where
  \[\lambda(n)=\sum_{d|P(n)}\mu(d)g(d),g(d)=\begin{cases}
  \inv{(k+l)!}(\ln\frac{D}{d})^{k+l}, & d<D,\\
  0, & d\geq D,
  \end{cases}l>0.\]

GPY gave $S_1\sim T_1x, S_2\sim T_2 x\ln x+O(E)$,
  \[T_1=\sum_{d_0d_1d_2}\frac{\mu(d_1d_2)g(d_0d_1d_2)}{d_0d_1d_2}g(d_0d_1)g(d_0d_2), T_2=\cdots.\]
\end{proof}

\end{document}
