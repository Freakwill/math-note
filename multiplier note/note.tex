\def\CTeXPreproc{Created by ctex v0.2.13, don't edit!}
\documentclass[11pt,a4paper,fleqn]{article}
\usepackage{mathrsfs}
\usepackage{amsfonts}
\usepackage{amsmath}
\usepackage{harmonic,analysis}%mypackage
\usepackage{amssymb}
\usepackage{eepic,epic}
\usepackage{enumerate}
\usepackage{graphicx,subfigure,float}
\usepackage{fancyhdr,times}
\usepackage{indentfirst}
\usepackage{hyperref}
%\usepackage{cite}
\usepackage[numbers,sort&compress]{natbib}

\hypersetup{
   colorlinks=true,%
   citecolor=green,%
   filecolor=black,%
   linkcolor=black,%
   urlcolor=blue
} % ��ӡ��ʱ��ע��

\begin{document}

\renewcommand{\S}{\mathscr{S}}                       % {x_n}
\renewcommand{\d}{\mathrm{d}}                          %
\renewcommand{\i}{\mathrm{i}}                          %
\renewcommand{\l}{\ell^1}
\newcommand{\ei}[1]{\mathrm{e}^{\mathrm{i}#1}}
\newcommand{\e}[1]{\mathrm{e}^{2\pi\mathrm{i}#1}}
\newcommand{\ec}[1]{\mathrm{e}^{-2\pi\mathrm{i}#1}}              %
\newcommand{\Mx}[2]{M^{s}_{p,q}}                         % ����ʹ��
\newcommand{\biggnorm}[1]{\bigg\|#1\bigg\|}                    %%norm
%\newcommand{\bignorm}[1]{\big\|#1\big\|}                                    %
\renewcommand{\M}{\mathcal{M}}   
\newcommand{\D}{\mathcal{D}}                             %
\renewcommand{\MM}{\mathfrak{M}}

\newenvironment{proof}{\emph{Proof.}\hspace{0.5cm}}{$\hspace{0.4cm}\Box$\\}
\newtheorem{theorem}{Theorem}[section]
\newtheorem{lemma}{Lemma}[section]
\newtheorem{remark}{Remark}[section]
\newtheorem{corollary}{Corollary}[section]
\newtheorem{fact}{Fact}[section]
\newtheorem{example}{Example}[section]
\newtheorem{definition}{Definition}[section]

%\footnotetext
\bibliographystyle{plain}  % ������˳������


\title{Note for Fourier Multipliers}
\date{}
\author{S. William}
\maketitle

\section{Lemmata of Inequality}
\begin{lemma}Assume $f\in \h{m,p}(\R^n)$,
\[\norm{\diff^m_{h}f}_{L^p} \lesssim |h|^m\norm{f}_{\h{m,p}}.\]
\end{lemma}
\begin{proof}Take case $m=1$ for example.
  Set an auxiliary function
  \[F(t)=\frac{\d f(x+th)}{\d t}=\nabla f(x+th)\cdot h\]
  for any $x,h\in\R^n$. We complete the proof, integrating $F$ form 0 to 1.
\end{proof}
The lemma bridges difference and  differential calculus (see Miao's book and \cite{triebel}). It appears in the proof of the equivalency of the norms of the function spaces.

Sequence space $\ell^{s,q}$ is defined as $\{\norm{a}=(\sum_k\jb{k}^{sq}|a_k|^q)^{\inv{q}}<\infty\}$ in the context of modulation spaces.
\begin{lemma}(see \cite{tomita})for any $R>0$,
\[(\int_{|x|>R}{|K|^p})^{\inv{p}}\lesssim\jb{R}^{-s}\norm{K}_{W(L^p,\ell^{s,1})},s\geq0.\]
\end{lemma}
\begin{proof}
Let $\Delta_k=\supp\phi_k$ and $B_R$ is the ball with radius $R$ about the origin.
Obviously $\Delta_k\cap B_R\neq\emptyset$ implies $|k|>\frac{R}{2}$ for large $R$. Thus
  \begin{eqnarray}
    (\intm{|K|^p}{|x|>R}{x})^{\inv{p}}&\leq&\sum_k(\intm{|\phi(x-k)K(x)|^p}{|x|>R}{x})^{\inv{p}}\nonumber\\
    &\lesssim&\sum_{|k|>R/2}\norm{\phi_k(x)K(x)}_p\label{key}\\
    &=&\sum_{|k|>R/2}\jb{k}^{-s}\jb{k}^{s}\norm{\phi_k(x)K(x)}_p\nonumber\\
    &\lesssim&\jb{R}^{-s}\sum_{k}\jb{k}^{s}\norm{\phi_k(x)K(x)}_p\nonumber\\
    &=&\jb{R}^{-s}\norm{K}_{W(L^p,\ell^{s,1})}.\nonumber
  \end{eqnarray}
  Since
  \begin{eqnarray}\label{W}
  &&W(L^p,\ell^1)\embed L^p,
  \end{eqnarray}
  for finite $R$ bounded by a certain constant,
  \begin{eqnarray*}
    (\intm{|K|^p}{|x|>R}{x})^{\inv{p}}&\leq&\norm{K}_p\\%   &\lesssim&\sum_{k}\norm{\phi_kK}_p\\
    &\leq&\norm{K}_{W(L^p,\ell^{1})}.
  \end{eqnarray*}
 Since $W(L^p,\ell^{s,1})\embed W(L^p,\ell^1)$, we complete the proof.
\end{proof}
\begin{remark}
One of more general forms of (\ref{W}) is
\[W(X,\ell^1)\embed X,\]
where $X$ is Banach function space.
\end{remark}
\begin{remark}$X^{s,p}$ denotes the function space with the norm
\[\sup_{R>0}\jb{R}^s(\int_{|x|>R}{|K|^p})^{\inv{p}},s\geq 0,\]
that is extremely like Murrey spaces. We have
\[W(L^p,\ell^{s,1})\embed X^{s,p}.\]
\end{remark}
\begin{remark}The key of the proof is (\ref{key}). It is true for any unitary partition consisting of compactly supported functions $\{\psi_k\}$ and the distances between supports $\Delta_k=\supp \psi_k$ and the origin tents to infinity as $|k|\rightarrow\infty$.
\end{remark}

\begin{corollary}
  \[M^s_{2,1}\embed \F X^{s,1}.\]
\end{corollary}

With same method, for diadic decomposition (see \cite{triebel}) we have
\begin{lemma}for any $r>0$,
\[(\int_{|x|>2^r}{|K|^p})^{\inv{p}}\lesssim 2^{-rs}\norm{K}_{W(L^p,\ell^{s,1})},s\geq0.\]
\end{lemma}
\begin{remark}
In the fact, it is just \[W(L^p,\ell^{s,1})\embed X^{s,p},\]
where $\ell^{s,q}$ is the sequence space with norm $(\sum_k2^{ksq}|a_k|^q)^{\inv{q}}$.
\end{remark}
\begin{corollary}
  \[\dot{B}_{2,1}^{s}\embed \F X^{s,1}.\]
\end{corollary}

\begin{fact}
 $\MM^1=\F M_{1,\infty}\embed \MM^p \embed \F M_{p,\infty}$.
\end{fact}
Notice $\MM^p=W(\mathcal{M}^p,\ell^{\infty})$. (see \cite{miyachi})

\begin{fact}
 $1\in M^{s}_{\infty,1},\F M_{1,\infty}$.
\end{fact}

\section{representation of Multipliers}
$\mathcal{D}(X,Y)$ represents the spaces of bounded operators $Tf(x)=d(x)f(x):X\rightarrow Y$, so $\mathcal{D}(X,Y)=\mathcal{M}(\F X,\F Y)$.
\begin{theorem}If $q_1\leq q_2$, then
  \[\mathcal{M}(M(X_1,\ell^{s_1,q_1}),M(X_2,\ell^{s_2,q_2}))= W(\mathcal{M}(X_1,X_2),\ell^{s_2-s_1,\infty})\]
  where $\norm{\psi_k f}_{X_i}\lesssim\norm{f}_{X_i}$.
Similarly,
  \[\mathcal{D}(W(X_1,\ell^{s_1,q_1}),W(X_2,\ell^{s_2,q_2}))= W(\mathcal{D}(X_1,X_2),\ell^{s_2-s_1,\infty}),\]
  where $X_1,X_2$ are solid.
\end{theorem}

\bibliography{multiplier}
\end{document}
